\documentclass[11pt,a4paper]{article}
\usepackage[utf8]{inputenc}
\usepackage[T1]{fontenc}
\usepackage[italian]{babel}
\usepackage{amsmath}
\usepackage{amsfonts}
\usepackage{amssymb}
\usepackage{graphicx}
\usepackage[backend=biber]{biblatex}
\addbibresource{biblio.bib}

\author{Luca Parolari}
\title{``Principi dell'Informatica'': il Corso Necessario nelle Scuole di Secondo Grado}

\begin{document}

\maketitle

\section{Introduzione}

Il lavoro del professore, in contrasto all'usuale immaginario
collettivo, è piuttosto arduo e prevede, oltre alle lezioni frontali,
un sacco di lavoro extra per la preparazione del materiale,
l'aggiornamento costante delle conoscenze, la correzione di compiti e
verifiche, la partecipazione agli impegni collegiali e molte
micro-attività che sul lungo periodo dimostrano (o confutano) la
qualità dell'insegnamento. Le lezioni frontali sono di fatto la punta
dell'iceberg di quello che un professore svolge, la parte più
difficile è probabilmente l'organizzazione dell'insegnamento, del
materiale e delle attività didattiche.

Questo fenomeno è molto più sentito nell'ambito informatico per una
serie di ragioni. L'informatica è una scienza piuttosto recente,
almeno dal punto di vista della percezione delle persone, e per questo
anche meno accettata. Disporre dell'insegnamento di informatica in un
corso di studi viene visto come qualcosa che viene aggiunto
sacrificando ore preziose di altre materie più importanti, portando di
fatto la concezione dell'informatica a qualcosa di non necessario e
strano.  L'informatica non è ancora vista come una materia vera e
propria, tanto che nelle scuole primarie di primo e secondo grado sono
pochissimi i casi in cui viene insegnata in modo serio e concreto.
Tutto ciò si ripercuote sulle scuole secondarie di secondo grado dove
invece esiste un insegnamento di informatica. \`E notoriamente più
difficile reperire dei programmi didattici ben fatti e assestati per
la materia informatica rispetto a, per esempio, italiano o
storia. Come conseguenza indiretta, risulta solitamente difficile
riuscire a trovare dei buoni manuali e testi di riferimento che siano
ben bilanciati e di aiuto ai discenti, i quali hanno bisogno di un
supporto per lo studio. Purtroppo, si dimostra molto difficile anche
trovare bravi docenti che siano in grado di insegnare dignitosamente
informatica. Questo capita non per la cattiva volontà delle persone,
ma per una combinazione di motivi che rendono il tutto infattibile:
nella nostra realtà chi ha conoscenze in informatica trova terreno più
fertile fuori dall'ambito didattico, le aziende infatti hanno bisogno
di figure esperte in materia e solitamente sono disposte a pagarle di
più rispetto al compenso offerto come docente. Inoltre, i docenti di
informatica sono spesso liberi professionisti che si improvvisano
professori, e l'improvvisazione comporta numerose problematiche dal
punto di vista didattico.

Tutto ciò è riassunto con un inevitabile disastro didattico per quanto
riguarda la materia informatica e sia i ragazzi che i professori si
adattano a questa situazione di sconforto generale, lasciando che
l'apprendimento una materia così importante al giorno d'oggi sia
lasciato al caso.

\section{Motivazione}

Qual è la motivazione e perché bisognerebbe passare ad un'altra
visione dell'informatica, introducendola a tutti i livelli nelle
scuole e cambiando radicalmente il sistema di istruzione?

Nel mondo odierno i computer sono dovunque e costituiscono un potente
strumento di aiuto per le persone. Per essere culturalmente preparato
a qualunque lavoro uno studente di adesso vorrà fare da grande è
indispensabile quindi una comprensione dei concetti di base
dell'informatica. Esattamente com'è accaduto in passato per la
matematica, la fisica, la biologia e la chimica. Il lato
scientifico-culturale dell'informatica, definito anche pensiero
computazionale, aiuta a sviluppare competenze logiche e capacità di
risolvere problemi in modo creativo ed efficiente, qualità che sono
importanti per tutti i futuri cittadini. L'essenza del concetto è che
con il pensiero computazionale si definiscono procedure che vengono
poi attuate da un esecutore (agente), che opera in modo meccanico e
inconsapevole nell'ambito di un contesto prefissato, per raggiungere
degli obiettivi assegnati. \`E importante ribadire che l'agente esegue
istruzioni (di cui però non conosce il significato), per elaborare
dati (di cui però non conosce il significato). In tal modo
un'elaborazione meccanica e inconsapevole riesce a replicare funzioni
cognitive umane.

Questi concetti sono stati riassunti in sette punti detti
\emph{Concetti Fondamentali} o \emph{Big Ideas} dell'informatica.
\begin{enumerate}
    \item La Creatività e l'informatica sono importanti forze
      propulsive dell'innovazione. Le innovazioni rese possibili
      dall'informatica hanno avuto e continueranno ad avere impatti
      estesi e di lunga durata.
    \item L'Astrazione riduce la quantità di dati e di dettagli da
      trattare, facilitando la concentrazione sugli aspetti più
      rilevanti. Si tratta di un processo, di una strategia, che
      permettono di meglio comprendere e risolvere un problema.
    \item I dati sono essenziali per la creazione di
      conoscenza. L'informatica realizza metodi efficienti per la loro
      elaborazione, rendendo così possibili cambiamenti impressionanti
      in tutte le discipline, dall'arte all'economia alla scienza.
    \item Gli Algoritmi vengono usati per sviluppare efficienti
      soluzioni operative a problemi risolvibili mediante elaborazione
      di dati. Espressi nei programmi informatici hanno cambiato il
      mondo in modo profondo e durevole.
    \item La Programmazione rende possibile la risoluzione dei
      problemi, la creazione di conoscenza e l'espressione umana. I
      programmi danno luogo a sistemi e strumenti informatici che
      facilitano la creazione di artefatti digitali, quali musica,
      immagini, visualizzazioni.
    \item Internet pervade il moderno panorama digitale: la sua
      diffusione ha cambiato radicalmente la società negli ultimi
      vent'anni, rendendo possibili modalità di comunicazione,
      interazione e collaborazione del tutto nuove.
    \item L'Impatto Globale dell'informatica è sotto gli occhi di
      tutti. Le interazioni sociali, il mondo degli affari e della
      produzione, la risoluzione dei problemi, sono stati cambiati
      dalle innovazioni dell'informatica, e cambieranno ancora.
\end{enumerate}
I concetti qui sopra riportati sono il manifesto di quello che il
pensiero computazionale ci vuole insegnare e possono essere visti come
degli obiettivi da raggiungere durante l'apprendimento del pensiero
computazionale: se uno comprende e maneggia questi sette concetti
allora probabilmente ha anche imparato a pensare come un informatico.

Le sette \emph{Big Ideas} sono spesso accompagnata da sei competenze
del pensiero computazionale, che rappresentano il saper fare di un
informatico.

\begin{enumerate}
    \item La capacità di connettere e di mettere in relazione i
      differenti concetti dell'informatica.
    \item La competenza nel creare artefatti computazionali permette
      di sviluppare creatività, lavorando alla progettazione e
      sviluppo di artefatti digitali e risolvendo problemi mediante
      l'utilizzo di tecniche informatiche.
    \item La capacità di astrarre per definire modelli e simulazioni
      di fenomeni naturali e artificiali, fare predizioni sulla loro
      evoluzione ed analizzarne efficacia e validità.
    \item L'abilità di analizzare problemi ed artefatti, sia
      realizzati in prima persona che sviluppati dagli altri:
      essenziale affinché ci sia progresso nella capacità di
      risoluzione dei problemi.
    \item L'abilità di comunicare è essenziale sia per spiegare e
      giustificare scelte progettuali e realizzative degli strumenti
      informatici, sia per analizzare e valutare i risultati ottenuti
      alla scelta delle esigenze iniziali.
    \item La capacità di collaborare sia nelle attività investigative
      relative ai dati ed alle loro relazioni che in quelle
      progettuali relative alla realizzazione degli artefatti
      digitali.
\end{enumerate}

\section{Come fare?}

Anche se le motivazioni ed i propositi fossero buoni, rimane
un'importante domanda aperta: come fare?

Varie organizzazioni in tutto il mondo si sono adoperate per sopperire
alla mancanza di un corso che insegni il pensiero computazionale. In
questa trattazione verrà preso in esame il corso ``Principi
dell'Informatica'' messo a disposizione dal gruppo di \emph{Programma
  Il Futuro}. Programma il Futuro è un progetto italiano ideato nel
2014 da Enrico Enrico Nardelli (CINI - Università di Roma Tor Vergata)
e Giorgio Ventre (CINI - Università di Napoli Federico II) in
collaborazione con il MIUR. Il progetto mira a fornire alle scuole una
serie di strumenti semplici, divertenti e facilmente accessibili per
formare gli studenti ai concetti di base dell'informatica.

Il ``parco strumenti'' di Programma il Futuro è molto ricco: sono offerti
corsi per le scuole primarie e secondarie, approfondimenti, seminari
sulla cittadinanza digitale ed eventi come l'ora del codice che
consento di studiare temi molto importanti dal punto di vista umano
e affrontarli con strumenti informatici.

Di seguito verrà analizzato il corso esplorandone punti forti e punti
deboli, cerando di capirne la struttura e l'organizzazione in modo
da capire se possa essere un valido candidato per insegnare
informatica agli studenti delle superiori.

\section{Il corso}

Per incominciare, il corso è rivolto a studenti delle scuole
secondarie di secondo grado: sono quindi i ragazzi dai quattordici ai
diciannove anni che si approcciano al pensiero computazionale. Ritengo
che sia di estrema utilità introdurre un corso come questo nelle
scuole superiori, in quanto ci sono ragazzi con molta voglia di fare,
ma spesso demotivati. Avere a disposizione il mondo dell'informatica
per costruire qualcosa di utile con le loro mani penso sia il modo
migliore per motivarli. Inoltre i ragazzi in questa fase della loro
vita sono chiamati ad entrare in contatto con il mondo del lavoro:
farlo con consapevolezza dell'informatica è di ovvia utilità sia per
comprendere al meglio i processi, le scelte, il lato nascosto di ciò
che vedono, ma anche per avere una skill in più nel loro bagaglio
delle competenze.

Il corso non ha requisiti particolari e ritengo ciò un grande
vantaggio. Per come stanno adesso le cose è impossibile pensare di
creare un corso basato su delle conoscenze base: l'introduzione
ufficiale di un insegnamento di informatica alle scuole primarie è
ancora futuro. Inoltre, il corso mette a disposizione moltissimo
materiale di supporto sia per i docenti che per i discenti nuovi per
la materia come attività didattiche,
video, documenti... tutti strumenti attraenti ed accessibili a
studenti e docenti delle più varie provenienze, esperienze, ed
interessi. Queste risorse sono state realizzate in modo da permettere
al docente di concentrarsi sul ruolo di facilitatore e tutore per lo
studente piuttosto che sul presentare ed illustrare direttamente i
vari concetti. In più il corso promette di poter essere usufruito
anche da docenti senza esperienza in informatica purché si trovino a
proprio agio con la tecnologia e abbiano un minimo di esperienza di
programmazione. Secondo me, bisognerebbe comunque cercare di preferire
docenti con formazione informatica in quanto penso che possano offrire
più spunti di approfondimento dare un'idea migliore su ciò che si sta
svolgendo.

\subsection{Struttura del corso}

Il corso si svolge lungo un percorso narrativo basato su Internet e
l'innovazione, come temi che connettono tutte le sei unità del corso
stesso. Si inizia da come inviare un bit da un posto ad un altro e si
finisce col riflettere sulle implicazioni di un'innovazione digitale.

Ognuna delle prime cinque unità sviluppa una storia relativa ad
uno specifico tema dell'informatica, partendo da un primo accenno fino
ad un più articolato sviluppo conclusivo. In particolare, le prime tre
unità sono dedicate a concetti fondamentali che sono specifico oggetto
di studio del corso come Internet, i dati, gli algoritmi e la
programmazione. La quarta unità sviluppa l'argomento dei big data e
del loro impatto globale sulla società, mentre la quinta unità
approfondisce il tema della programmazione per lo sviluppo di
applicazioni. La sesta unità è dedicata ad attività progettuale di
preparazione all'esame finale.

Le lezioni sono state realizzate in modo da essere centrate sullo
studente che viene stimolato a procedere con attività basate su
indagini finalizzate a scoprire e comprendere i concetti. Non è quindi
necessario che l'insegnante svolga una tradizionale lezione frontale.
Le istruzioni dirette per l'uso del materiale
didattico sono contenute all'interno del materiale stesso. Ciò secondo
me è molto importante: la scuola di oggi è quasi tutta basata su
lezioni dove lo studente partecipa in modo passivo. Questa passività
tende spesso ad annoiare lo studente riducendo drasticamente
l'efficacia della lezione e ciò si ripercuote sulla vita dello
studente.

Inoltre, il corso è stato pensato particolarmente adatto ad
essere inserito anche all'interno di un'organizzazione didattica
normalmente rigida quale quella esistente, con gli attuali orari della
scuola secondaria superiore. È infatti possibile dispiegare lo
svolgimento del corso (che richiede complessivamente agli studenti tra
le 150 e le 200 ore) sia lungo un solo anno scolastico che durante più
anni, in funzione del numero di ore settimanali che si riesce ad
assegnargli.

\subsection{Struttura delle lezioni}

Le lezioni sono quasi tutte strutturate in tre fasi differenti. La
prima fase, di introduzione, consente allo studente di capire che cosa
sa su quell'argomento e se le sue conoscenze sono giuste o errate. Lo
studente in questa fase riesce a capire intuitivamente qual è il
concetto di cui si parla e relative problematiche. Viene poi una
seconda fase più sostanziale, dove lo studente è chiamato a lavorare
in modo attivo per risolvere i problemi trovati in precedenza o per
creare qualcosa di concreto. Viene solitamente favorita la
collaborazione e le soluzioni di gruppo. Segue poi una fase conclusiva
in cui si effettua una sintesi dell'argomento o una riflessione. Lo
studente può quindi sistemare le sue idee e capire a fondo le scelte e
le soluzioni adottate nella fase precedente, proponendo anche
miglioramenti alla precedente soluzione.

L'approccio delle lezioni è sempre centrato sullo studente: lo
studente partecipa attivamente alla lezione viene stimolato a
ragionare e produrre idee che siano critiche, sappiano risolvere un
problema e lo sappiano fare in modo computazionale. Allo studente sono
spesso forniti strumenti a supporto delle soluzioni sviluppate. Il
fatto che questi strumenti siano di tipo informatico (ad esempio, il
simulatore di internet, AppLab per la creazione di app) obbligano lo
studente ad inventare una soluzione che risolva si il problema, ma che
sia anche eseguibile da un elaboratore, obiettivo del pensiero
computazionale.

\subsection{Le unità didattiche}

\paragraph{Internet}

Questa unità, divisa in due capitoli, vuole esplorare i problemi e le
sfide tecniche che derivano dalla necessità di rappresentare i dati in
formato digitale per consentire al computer la loro elaborazione e
trasmissione.

Nel primo capitolo si inizia analizzando cosa è necessario per inviare
un dato elementare (ad esempio, il bit) da un posto ad un altro. Gli
studenti poi sono chiamati a lavorare sulle problematiche dell'invio
di dati con più bit e maggiore struttura, assistendo così a un aumento
della complessità. Durante le lezioni viene utilizzato il Simulatore
di Internet, che permette agli studenti di sperimentare direttamente
sul campo le codifiche binarie dei dati e i protocolli di
comunicazione.

Gli studenti poi proseguono affrontando lo stesso tipo di problemi che
sono stati risolti nella costruzione di Internet e sviluppando i loro
protocolli di comunicazione, ciascuno basato sul precedente, ed
arrivando ad un loro stratificazione, come accade nella rete reale.

Il progetto svolto nell'unità richiede allo studente di investigare un
problema sociale attuale legato ad Internet quale, ad esempio, la
neutralità della rete o la censura sulla rete.

\paragraph{Dati digitali}
Questa unità approfondisce il tema della rappresentazione ed
elaborazione digitale dei dati. Gli studenti imparano a generare,
ripulire, ed elaborare dati, utilizzando poi strumenti di
visualizzazione per esplorare ed identificare tendenze e schemi
ricorrenti.

L'unità riparte dalla rappresentazione binaria dei dati ed analizza
varie tecniche utili per risparmiare lo spazio necessario per la
rappresentazione di testo ed immagini (tecniche di compressione). Un
primo progetto richiede agli studenti di definire il loro schema di
compressione dei dati per una loro esperienza significativamente
complessa.

Successivamente, gli studenti si esercitano nell'interpretazione di
dati visuali e nell'utilizzo di strumenti per creare i loro personali
artefatti digitali. In questo processo, imparano a raccogliere e
ripulire i dati, e ad utilizzare strumenti per aggregarli e
visualizzarli. Riflettono inoltre sul ruolo dell'errore umano durante
la raccolta e l'analisi e di come questo possa portare a conclusioni
inaccurate e potenzialmente dannose.

Nel progetto finale gli studenti devono lavorare in collaborazione su
un insieme di dati grezzi, usando strumenti per individuare possibili
relazioni e tendenze, realizzare un spiegazione visuale di quanto
hanno trovato, e scrivere un documento illustrativo.

\paragraph{Algoritmi e programmazione}

Questa unità introduce la necessità degli algoritmi ed avvia lo
studente al linguaggio di programmazione JavaScript, insegnandogli a
sviluppare le prime semplici applicazioni fruibili attraverso il Web
(le app).

Si inizia con attività tradizionali (senza utilizzare i PC) per
introdurre gli algoritmi e mostrare la necessità di un linguaggio di
programmazione che renda possibile eseguirli su un PC. In questo
percorso si stabiliscono connessioni con le regole definite nei
protocolli per Internet discussi nella prima unità, nei quali erano
stati gli studenti ad agire come computer. Questo permette di
comprendere come l'essenza di un programma informatico sia una serie
di passi che una macchina digitale è in grado di compiere
autonomamente e meccanicamente.

Gli studenti vengono introdotti all'uso dell'ambiente di
programmazione \emph{AppLab}, in cui scrivono programmi che
controllano una tartaruga, personaggio virtuale che si muove sullo
schermo e realizza disegni. Lo studio di JavaScript è centrato sullo
sviluppo di procedure (o funzioni, cioè porzioni di programmi
riutilizzabili) in modo da abituare lo studente ad un approccio
top-down nella realizzazione dei programmi.

Col progetto finale gli studenti collaborano per progettare e
condividere codice che crea, mediante la tartaruga, una storiella
disegnata.

\paragraph{Big Data e privacy}

In questa unità gli studenti sviluppano una visione complessiva e
bilanciata sui dati nel mondo intorno a loro e sui loro effetti
positivi e negativi. Inoltre, studiano le basi del funzionamento delle
moderne tecniche crittografiche.

L'unità inizia facendo riflettere gli studenti sull'enorme quantità di
dati digitali (big data) del mondo moderno, sul ruolo essenziale che
hanno i computer nella loro elaborazione, e sui possibili usi sia
positivi (ad esempio, per sostenere l'innovazione) che negativi (ad
esempio, per controllare e censurare le persone).

Si prosegue con lo studio della crittografia, partendo da un'attività
di tipo tradizionale (senza computer) e proseguendo con gli strumenti
crittografici forniti per esplorare ulteriormente il concetto. In tal
modo gli studenti riescono a comprendere le varie idee matematiche
alla base dello sviluppo delle tecniche di crittografia e come/quando
tali tecniche possono essere violate. In questo processo, acquisiscono
consapevolezza dell'esistenza di problemi computazionalmente
difficili. Alla fine, lo studente acquisisce una conoscenza di alto
livello della crittografia a chiave asimmetrica, di cosa essa rende
possibile (inviare dati criptati senza condividere una chiave) e di
cosa rende sostanzialmente impossibile (violare i dati criptati).

Il progetto finale è focalizzato su di un argomento di attualità
relativo ai big data, la sicurezza e la crittografia.

\paragraph{Realizzazione di App}

Questa unità approfondisce lo studio del linguaggio di programmazione
JavaScript e lo studente impara a realizzare app guidate dagli
eventi. L'unità presuppone l'acquisizione di concetti e competenze
trattati nell'unità 3 (Algoritmi e programmazione): scrivere ed usare
le funzioni, l'utilizzo dei cicli, leggere la documentazione, usare
l'ambiente AppLab, la collaborazione.

Si comincia usando AppLab in modalità di progettazione, la quale
permette agli studenti di prototipare rapidamente un'app basata su
eventi. Man mano che gli studenti realizzano applicazioni che
reagiscono alle azioni degli utenti, vengono esaminati ed acquisiti
concetti chiave della programmazione: le variabili, la logica
booleana, le istruzioni condizionali.

Si prosegue approfondendo ulteriori concetti chiave, quali i cicli, i
vettori, le simulazioni. Gli studenti imparano a valutare la
correttezza di un programma e le modalità di realizzazione che rendono
più facile tale compito e la futura evoluzione del programma stesso.

L'unità si conclude con un progetto nel quale si riprende una delle
app sviluppate nel corso dell'unità 3 e si identificano, progettano e
realizzano possibili miglioramenti.

\paragraph{Prova finale}

L'unità è dedicata alla realizzazione da parte dello studente dei due
progetti finali. Un progetto è dedicato alla creazione di un programma
informatico, mentre l'altro è dedicato all'esplorazione di
un'innovazione dell'informatica.

La realizzazione di questa unità dovrà essere modificata da ogni
docente per adattarla alla specifica situazione organizzativa della
sua classe/scuola. Il coinvolgimento del docente è minimale e
focalizzato essenzialmente ad assicurare che lo studente abbia
obiettivi ben chiari e piani realizzativi accettabili.

Il progetto di creazione deve avere dimensioni tali da poter essere
realizzato in autonomia dallo studente in circa 12 ore, mentre quello
di esplorazione dovrà richiedere allo studente di lavorare in
autonomia per circa 8 ore.

\section{Conclusione}

In conclusione, ritengo che il corso proposto sia un valido candidato
per l'insegnamento di informatica in tutte le scuole superiori, sia 
professionali, tecnici, licei. Il corso è, secondo me, strutturato
in mondo molto moderno e ha molte potenzialità. Da subito è possibile
notare come questo corso cerca di rendere partecipe lo studente in
prima persona e stimolarlo a non essere più solo un ascoltatore. 
Ciò consente di sviluppare il senso critico, il pensiero, le idee 
di ogni studente.
Lo studente inoltre riesce ad acquisire tutte le conoscenze base in
campo informatico, cosa impossibile con gli attuali ministeriali,
spesso antiquati e attaccati al tradizionale schema di insegnamento.
Altro punto di forza di questo corso è che cerca di estirpare un 
altro grosso problema, legato stavolta alle conoscenze del docente: 
capita infatti troppo spesso che il programma di informatica sia
adattato in base alle conoscenze del docente invece che a quelle 
della classe perché, ad esempio, il docente conosce solo il 
linguaggio X invece che il linguaggio Y. Il materiale a corredo di
\emph{Principi di Informatica} invece azzera questo problema dando 
la possibilità a tutti, con un piccolo sforzo, di entrare in questo
mondo.

Sicuramente questo corso rappresenta una novità per tutti, le novità sono cose sconosciute e spesso le cose sconosciute spaventano, tendono a rendere l'essere umano statico e sedentario. Penso sia però arrivato il tempo di uscire dalla comfort zone e fare il passo perché i nostri studenti hanno bisogno di questo.

\nocite{CodeOrg} \nocite{ProgrammaIlFuturo} \nocite{BJC} \nocite{MichaelLodi} \nocite{EnricoNardelli}
\printbibliography

\end{document}
