%%%%%%%%%%%%%%%%%%%%%%%%%%%%%%%%%%%%%%%%%
% Beamer Presentation
% LaTeX Template
% Version 1.0 (10/11/12)
%
% This template has been downloaded from:
% http://www.LaTeXTemplates.com
%
% License:
% CC BY-NC-SA 3.0 (http://creativecommons.org/licenses/by-nc-sa/3.0/)
%
%%%%%%%%%%%%%%%%%%%%%%%%%%%%%%%%%%%%%%%%%
% Personalization by Luca Parolari <luca.parolari23@gmail.com>

\documentclass[
    hyperref={colorlinks,citecolor=black,linkcolor=black,urlcolor=blue}
]{beamer}

%-------------------------------------------------------------------
%	PACKAGES AND THEMES
%-------------------------------------------------------------------

%%% PACKAGES
\usepackage[utf8]{inputenc}  % italian symbols.
\usepackage[T1]{fontenc}     % define T1 charset for out files.
\usepackage[italian]{babel}  % italian latex typo conventions.
\usepackage{csquotes}        % needed by babel.
\usepackage{amsmath}         % math features.
\usepackage{amsthm}          % math theorems.
\usepackage{amssymb}         % math symbols.
\usepackage{graphicx}        % images managing.
\usepackage{booktabs}        % Allows the use of \toprule, \midrule and \bottomrule in tables
\usepackage{algorithm}       % algorithm block.
\usepackage{algcompatible}
\usepackage{algpseudocode}   % style for (autoimported) package algorithmicx.

%%% MODE
\mode<presentation> {
% The Beamer class comes with a number of default slide themes
% which change the colors and layouts of slides. Below this is a
% list of all the themes, uncomment each in turn to see what they
% look like.

%\usetheme{default}
%\usetheme{AnnArbor}
%\usetheme{Antibes}
%\usetheme{Bergen}        % commented left
%\usetheme{Berkeley}      % menu left
%\usetheme{Berlin}
%\usetheme{Boadilla}      % nice, no top menu
%\usetheme{CambridgeUS}   % nice menu and footer
%\usetheme{Copenhagen}
%\usetheme{Darmstadt}
%\usetheme{Dresden}
%\usetheme{Frankfurt}     % top bullets, no info below
%\usetheme{Goettingen}    % no
%\usetheme{Hannover}      % no
%\usetheme{Ilmenau}       % not bad: bullets, sections, ... but too many rows
%\usetheme{JuanLesPins}   % tree menu
%\usetheme{Luebeck}       % very nice menu
\usetheme{Madrid}        % default
%\usetheme{Malmoe}
%\usetheme{Marburg}
%\usetheme{Montpellier}
%\usetheme{PaloAlto}
%\usetheme{Pittsburgh}    % clean
%\usetheme{Rochester}
%\usetheme{Singapore}     % not bad, no info below
%\usetheme{Szeged}
%\usetheme{Warsaw}

% As well as themes, the Beamer class has a number of color themes
% for any slide theme. Uncomment each of these in turn to see how it
% changes the colors of your current slide theme.

%\usecolortheme{albatross}     % horrible
%\usecolortheme{beaver}        % red blue
%\usecolortheme{beetle}        % horrible
\usecolortheme{crane}         % nice: yellow, orange
%\usecolortheme{dolphin}       % simil default
%\usecolortheme{dove}          % black and white
%\usecolortheme{fly}           % horrible
%\usecolortheme{lily}          % nice, clean blue
%\usecolortheme{orchid}        % like default
%\usecolortheme{rose}          % like default, very good
%\usecolortheme{seagull}       % grey
%\usecolortheme{seahorse}      % light lavanda
%\usecolortheme{whale}         % like default
%\usecolortheme{wolverine}     % yello blue

\usefonttheme{professionalfonts}

% To remove the footer line in all slides uncomment this line
%\setbeamertemplate{footline}

% To replace the footer line in all slides with a simple slide 
% count uncomment this line
%\setbeamertemplate{footline}[page number]

% To remove the navigation symbols from the bottom of all 
% slides uncomment this line
\setbeamertemplate{navigation symbols}{} 

% blocks
%\setbeamertemplate{blocks}[rounded][shadow=true]

} % /mode<presentation>

% top menu
\useoutertheme[subsection=false]{miniframes}

%\setbeamercolor{block title}{use=structure,fg=white,bg=blue!75!black}
%\setbeamercolor{block body}{use=structure,fg=black,bg=white!20!white}

% step by step
\setbeamercovered{transparent}


%-------------------------------------------------------------------
%	TITLE PAGE
%-------------------------------------------------------------------

% The short title appears at the bottom of every slide, 
% the full title is only on the title page
\title[
  Approfonfimento AALP
]{
    ``Principi dell'Informatica'':\\ il Corso Necessario nelle Scuole di Secondo Grado
}
%\subtitle{}

\author[Luca Parolari]{
  Luca Parolari
}

\institute[UNIPD]
{
  Università di Padova \\
  Dipartimento di Matematica \\
  Corso di Laurea Magistrale in Informatica
}
\date{Maggio 2020}



\begin{document}

%------------------------------------------------

\begin{frame}
  \titlepage
\end{frame}

%------------------------------------------------

\begin{frame}
  \frametitle{Panoramica della presentazione}
  \tableofcontents
\end{frame}

%-------------------------------------------------------------------
%	PRESENTATION SLIDES
%-------------------------------------------------------------------

%------------------------------------------------
\section{Introduzione}
%------------------------------------------------

\begin{frame}
  \frametitle{Le problematiche attuali} 
  %\framesubtitle{}

  \begin{itemize}
    \item L'informatica non è ancora socialmente accettata da tutti
    \item Conoscenze pregresse spesso inesistenti
    \item La qualità dei corsi di informatica attuali è scarsa
    \item Le modalità didattiche impiegate sono non interessanti e passive per lo studente
    \item Insegnanti spesso impreparati e/o non al passo con i tempi
  \end{itemize}
\end{frame}

%------------------------------------------------
\section{Motivazioni}
%------------------------------------------------

\begin{frame}
  \frametitle{Perché bisognerebbe cambiare le cose?} 

  \begin{itemize}
    \item Informatica ubiqua e necessaria nella società odierna
    \item Dal punto di vista scientifico-culturale, imparare l'informatica significa sviluppare competenze logiche e capacità di risolvere problemi in modo creativo ed efficiente
  \end{itemize}
\end{frame}

\begin{frame}
  \frametitle{Il pensiero computazionale}

  Saper pensare ``computazionalmente'' significa essere in grado di sviluppare soluzioni che siano però eseguibili da un agente.
  
  \vspace{1em}

  Inoltre, l'esecutore è inconsapevole ed
  \begin{itemize}
    \item esegue istruzioni di cui non conosce il significato
    \item su dati di cui non conosce il significato
  \end{itemize}
\end{frame}

\begin{frame}
  \frametitle{Le sette Big Ideas}

  Il pensiero computazionale è stato riassunto in sette concetti fondamentali.
  
  \begin{enumerate}
    \item La creatività, come forza propulsiva dell'innovazione
    \item L'astrazione, per concentrarsi su aspetti più rilevanti
    \item I dati, per la creazione di conoscenza
    \item Gli algoritmi, per sviluppare soluzioni efficienti
    \item La programmazione, per risolvere problemi
    \item Internet, per comunicare e collaborare
    \item L'impatto globale, da tenere in considerazione per ogni scelta in ambito informatico
  \end{enumerate}
\end{frame}

\begin{frame}
  \frametitle{Le sei competenze}

  Il pensiero computazionale comporta anche del ``saper fare'', riassunto in sei competenze.
  
  \begin{enumerate}
    \item Mettere in relazione i concetti dell'informatica
    \item Creare artefatti computazionali
    \item Definire modelli e simulazioni
    \item Analizzare problemi ed artefatti
    \item Comunicare i concetti
    \item Collaborare con altre persone
  \end{enumerate}
\end{frame}

%------------------------------------------------
\section{Come fare?}
%------------------------------------------------

\begin{frame}
  \frametitle{Lo stato dell'arte}

  Attualmente molte organizzazioni si sono mosse per creare un corso valido che possa sopperire a questa necessità, per esempio
  
  \begin{itemize}
    \item \emph{Code.org}, che offre una quantità smisurata di materiale per corsi adatti ad ogni fascia di età
    \item \emph{Programma il Futuro}, un progetto italiano simile a Code.org
    \item \emph{CINI} (Consorzio Interuniversitario Nazionale per l'Informatica), che coordina e promuove le attività scientifiche legate all'informatica
    \item Università di Berkeley con \emph{BJC} (the Beauty and the Joy of Computing), con un corso di informatica di base pre-universitario
  \end{itemize}
\end{frame}

%------------------------------------------------
\section{Il corso}
%------------------------------------------------

\begin{frame}
  \frametitle{Principi dell'Informatica}
  \framesubtitle{Caratteristiche}

  \begin{table}
  \begin{tabular}{l p{75mm}}
    \textbf{Target} & Scuola secondaria di secondo grado (età 14-19) \\
    \textbf{Prerequisiti} & Nessuno per gli studenti, conoscenze base di informatica per i professori \\
    \textbf{Tipo lezioni} & Collaborative e attive per lo studente \\
    \textbf{Supporto} & Materiale integrato nelle lezioni \\
    \textbf{Durata} & Circa 150-200 ore
  \end{tabular}
  \end{table}
\end{frame}

\begin{frame}
  \frametitle{Struttura del corso}

  \begin{itemize}
    \item Sviluppa un percorso narrativo basato su internet e innovazione per interconnettere tutti i temi
    \item Sei unità didattiche che trattano le sette \emph{Big Ideas}
    \item Organizzazione flessibile
    \item Prova alla fine del corso
  \end{itemize}

  Nota: per il docente è presente anche un documento che approfondisce il curriculum sotto vari aspetti tra cui quello pedagogico, filosofia del corso e pratica in classe.\footnote{\href{https://docs.google.com/document/d/1ZVzF_-cON8pjDVUOZjVk32y4flCMXugcrA6gFeWDHzE/preview}{Curriculum Guide}}
\end{frame}

\begin{frame}
  \frametitle{Struttura delle lezioni}

  \begin{itemize}
    \item Diverse dalle lezioni tradizionali
    \item Sviluppo lungo tre fasi (introduzione, sviluppo, conclusione)
    \item Favorisce la collaborazione di gruppo
    \item Approccio centrato sullo studente, lo studente partecipa in modo attivo
    \item Il materiale di studio è fornito direttamente con la lezione\footnote{Esempio: \href{https://curriculum.code.org/csp-19/unit1/1/}{Unit 1 Lesson 1: Personal Innovations}}
    \item Favorisce l'utilizzo di strumenti informatici per la comprensione dei concetti
  \end{itemize}
\end{frame}

\begin{frame}
  \frametitle{Unità didattiche}

  \begin{enumerate}
    \item Internet
    \item Dati digitali
    \item Algoritmi e programmazione
    \item Big Data e privacy
    \item Realizzazione di App
    \item Prova finale
  \end{enumerate}
\end{frame}

\begin{frame}
  \frametitle{UD: Internet}

  \begin{itemize}
    \item Punto di partenza, solitamente anche il più conosciuto da parte dello studente
    \item Esplorare problemi della rappresentazione dell'informazione in formato digitale
    \item Nel primo capitolo si affrontano le tematiche della trasmissione della rappresentazione tramite codifiche
    \item Nel secondo capitolo si affronta il problema della costruzione di internet, sviluppando protocolli e stratificandoli
    \item Attività pratiche
    \begin{itemize}
      \item Simulatore di internet
      \item Prototipazione di un modello semplificato di internet
    \end{itemize}
  \end{itemize}
\end{frame}

\begin{frame}
  \frametitle{UD: Dati digitali}

  \begin{itemize}
    \item Esplorare i problemi della rappresentazione ed elaborazione digitale dei dati
    \item Gli studenti imparano a rappresentare, pulire ed elaborare i dati. Vengono esplorate tecniche di compressione dei dati
    \item Analisi dei dati con strumenti di visione per individuarne pattern ricorrenti e tendenze
    \item Rilevanza dell'errore nei dati e conseguenti problematiche nell'inferenza di conoscenze
    \item Attività pratiche
    \begin{itemize}
      \item Costruzione di una propria tecnica di compressione dei dati
      \item Analisi di un insieme di dati grezzo e scrittura di un artefatto che spieghi cosa si è potuto inferire dai dati
    \end{itemize}
  \end{itemize}
\end{frame}

\begin{frame}
  \frametitle{UD: Algoritmi e programmazione}

  \begin{itemize}
    \item Esplorare il concetto di algoritmo e di programmazione
    \item Introduzione agli algoritmi e prima applicazioni in JavaScript
    \item Programma informatico come insieme di regole, analogia con i protocolli di Internet
    \item Introduzione ad \emph{AppLab} per la costruzzione di App JavaScript
    \item Realizzazione di programmi top-down e introduzione man mano degli strumenti del linguaggio di programmazione
    \item Attività pratica: costruzione di un'applicazione
  \end{itemize}
\end{frame}

\begin{frame}
  \frametitle{UD: Big Data e privacy}

  \begin{itemize}
    \item Esplorare l'importanza dei dati e le basi della crittografia moderna
    \item Riflessione sulla quantità di dati nel mondo moderno ed effetti positivi/negativi che possono avere sulla società
    \item Studio della crittografia applicata in modo da ottenere la privacy
    \item Introduzione all'idea del problema computazionalmente difficile
    \item Attività pratica sui big data e sicurezza con crittografia
  \end{itemize}
\end{frame}

\begin{frame}
  \frametitle{UD: Realizzazione App}

  \begin{itemize}
    \item Approfondire la programmazione tramite JavaScript e la programmazione per eventi
    \item Prerequisito: unità 3 (Algoritmi e programmazione)
    \item Approfondimento dei concetti base della programmazione (funzioni, cicli, leggere la documentazione, collaborare, ...)
    \item Costruzione di un'applicazione semplice ma completa
    \item Attività pratica con \emph{AppLab}
  \end{itemize}
\end{frame}

\begin{frame}
  \frametitle{UD: Prova finale}

  \begin{itemize}
    \item Due progetti conclusivi
    \item Creazione di un programma in autonomia (circa 12 ore)
    \item Studio di un'innovazione informatica (circa 8 ore)
  \end{itemize}

  \begin{itemize}
    \item L'unità deve essere modificata dal docente di modo che sia adattata alla situazione della classe
  \end{itemize}
\end{frame}

%------------------------------------------------
\section{Conclusioni}
%------------------------------------------------

\begin{frame}
  \frametitle{Conclusioni}

  \begin{itemize}
    \item Il corso ha moltissime potenzialità: moderno, ben strutturato, adatto a tutti
    \item Il corso è completamente slegato dalle conoscenze del professore
    \item Lo studente diventa attore della lezione, non un semplice ascoltatore passivo
  \end{itemize}
  \begin{itemize}
    \item Problema: attaccamento ``all'italiana'' alle abitudini e alla cultura
  \end{itemize}

  \vspace{1em}

  \centerline{Ce la faremo?}
\end{frame}

\begin{frame}
  \Huge{\centerline{Grazie per l'attenzione}}
\end{frame}

%-------------------------------------------------------------------

\end{document}
